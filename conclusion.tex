% ---------------------------------------------------
% ----- Conclusion of the template
% ----- for Bachelor-, Master thesis and class papers
% ---------------------------------------------------
%  Created by C. Müller-Birn on 2012-08-17, CC-BY-SA 3.0.
%  Freie Universität Berlin, Institute of Computer Science, Human Centered Computing. 
%
\chapter{Zusammenfassung}
\label{chap:conclusion}      

%\begin{itemize}
%	\item Die Zusammenfassung sollte das Ziel der Arbeit und die zentralen
	% Ergebnisse beschreiben. Des Weiteren sollten auch bestehende Probleme bei der Arbeit aufgezählt werden und Vorschläge herausgearbeitet werden, die helfen, diese Probleme zukünftig zu umgehen. Mögliche Erweiterungen für die umgesetzte Anwendung sollten hier auch beschrieben werden.
%\end{itemize}
Das Thema Ontology Matching hat in den letzten Jahren durch Machine bzw. Deep
Learning und die Verbreitung von IT in den meisten Wissenschaftsbereichen an
Relevanz gewonnen, da Ontologien es erlauben, Konzepte aus unterschiedlichsten
Domänen für Computer verarbeit- und verknüpfbar zu machen. Dabei
finden Technologien, wie OWL und RDF Anwendung, die definieren, wie man Daten
strukturieren kann. In einer solchen Modellierung lassen sich die Daten
auszutauschen und untereinander verknüpfen.\\
Der Fokus dieser Masterarbeit lag darauf, ein Tool zu entwickeln, das technisch
nicht-versierten Nutzern entgegen kommt und es ihnen ohne größeren Aufwand
ermöglicht, Ontologien zu matchen. Dabei sollen die Ergebnisse, die
vorgeschlagen werden, keine außergewöhnlich hohe Rate an false positives
aufweisen, wodurch es wenig nützlich wäre.\\
Aus Sicht der Softwareentwicklung soll es modular aufgebaut sein, um leicht
erweiter- und wartbar zu sein. Dadurch wäre es potenziell auch möglich, die
einzelnen Teile außerhalb der eigentlichen Software zu nutzen.\\
Da es sich um wissenschaftliche Arbeit handelt, ist auch sinnvoll, die
Ergebnisse frei verfügbar zu veröffentlichen, sowohl die schriftlichen Anteile
als auch die Software. Dadurch wird dem wissenschaftlichen Gedanken entsprochen,
mit einer wissenschaftlichen Arbeit Wissen zu mehren und zu teilen.

\section{Umsetzung}
Die Software wurde von Anfang an als Webapp entwickelt, damit die Nutzer es
einfach durch den Aufruf der Seite starten können. Die Arbeit, es aufzusetzen
und zu konfigurieren, wird dabei vom Nutzer auf einen Admin verschoben. Also von
einem nichttechnischen zu einem technischen Experten.\\
Zuerst wurden eine Wrapper-Schicht entwickelt, die das Front- vom Backend
trennt, aber strukturierte Zugriffe anbietet. Die Backend Module, konkret die
Reader und Matcher, werden dynamisch geladen. Das erlaubt es, diese Module erst
bei der Verwendung zu laden und es wird vermieden, sie vorher importieren bzw.
bekannt machen zu müssen. Weiterhin werden Ergebnisse und Daten zwischen den
verschiedenen Modulen ausgetauscht, ohne dass die Grundmodule Logik über die
Funktionsweise enthalten müssen. Diese beiden Punkte sorgen für hohe
Modularität.\\
Das Frontend wird durch ein Django View umgesetzt. Dort werden die
Nutzeranfragen entgegen genommen, verarbeitet und dann die Antwort
präsentiert. Das Frontend greift dabei auf die Wrapper zurück, um die Eingaben
entsprechend weiterzuleiten. Die konkrete Verarbeitung der Optionen wird dabei
von den Wrappern durchgeführt.\\
Die erste Art von Wrapper Modulen ist das \textit{Reader Modul}. Reader Module
lesen Ontologien ein und überführen diese Daten in strukturierte Formate.
Diese Datenformate bilden die Strukturen von Ontologien und deren Elementen ab
und erlauben es, problemlos Ontologien aus unterschiedlichen Quellen
gleichwertig bearbeiten zu können.\\
Die zweite Art von Wrappern ist das \textit{Matcher Modul}. Es nimmt die zu
matchenden Ontologien und reicht es an den ausgewählten Matcher weiter. Beides erhält
dieser Wrapper von einer \textit{Matching Tool Chain}, in einem
gleichnamigen Modul. Diese Tool Chain sammelt die Matcher, prüft ihre
Existenz und führt sie der Reihe nach aus.\\
Die implementierten Matcher decken verschiedene Methoden des Ontology Matchings
ab, die in dieser Arbeit vorgestellt wurden. Alle Matcher haben verschiedene
Iterationen bis zu ihrem jetzigen durchlaufen. Die Matcher sollen sowohl als
exemplarisches Beispiel für die Nutzung der Software als auch als konkrete
Werkzeuge für das Ontologie Matching dienen.\\
Die Matcher, die implementiert wurden, beschränken sich auf die Label der
einzelnen Elemente und vergleichen diese, um mögliche Verbindungen zwischen
diesen zu suchen. Zwei Ausnahmen sind der Same Entity und der Count Matcher.
Beim Same Entity Matcher werden die URIs der zwei Elemente auf Gleichheit
geprüft, um Elemente, die in beiden Ontologien vorkommen, zu finden. Der Count
Matcher sammelt und zählt die in unterschiedlichen Matcher gefundenen Paare. 
Der Gedanke dahinter war, dass wenn ähnliche Matcher ein Paar mehrfach finden,
die Wahrscheinlichkeit steigt, dass dieses korrekt ist. Die Idee dieses
Matcher entstand relativ früh im Entwicklungsprozess. Da die Matcher aber nun
relativ unterschiedliche Algorithmen haben, macht dieses Konzept aktuell nur
bedingt Sinn. Jetzt zeigt er aber praktisch am Beispiel immer noch, wie eine
mögliche Analyse der Ergebnisse anhand der bisherigen Matchings funktionieren kann.\\
Zeitlich hängen die Matcher zwar überwiegend von der Menge
der Eingabedaten, also der Ontologien, ab, die Verwendung von regulären Ausdrücken
zeigt sich trotzdem in der Dauer, die ein Durchlauf der Matcher benötigt. Dies
zu ändern, ist in der Praxis schwierig, sofern es keine von der Laufzeit
effizientere Implementierung von regulären Ausdrücken oder eine Implementierung
von Nebenläufigkeit gibt.
In der Hinsicht auf die Laufzeit
schneiden auch die Matcher, die PyDictionary verwenden, leider auch nicht
positiv ab.\\
Nach der Entwicklung des Backend folgte die Erstellung des Frontend und der
benötigten Templates für die Start- und Ergebnisseite. Die Startseite sollte
direkt die Auswahlmöglichkeiten zeigen und dem Nutzer nur das präsentieren, was
er für die unmittelbare Arbeit, das Matching von Ontologien benötigt. Nämlich
die Ontologien und die Matcher. Dadurch soll der Nutzer sich ohne Ablenkung
direkt der Aufgabe widmen kann. Anfangs wurden die Ontologien und Matcher
einfach nur aufgeführt, ohne konkrete Anweisungen. Das ist aber nicht
ausreichend, um einen klaren geführten Prozess abzubilden, den die Nutzer
einfach verstehen und anwenden können. Daher wurden Aufforderungen, die
entsprechenden Optionen anzuwählen, eingefügt.\\
Die Ontologien, die zur Auswahl stehen, werden beim Start der Django App
gesammelt. Welche Arten von Quellen möglich sind, hängt von den Readern, aber
auch von angegebenen Quellen ab. Aktuell werden lokale OWL Dateien und die
Terminologien des GFBio Terminology Servers als mögliche Quellen verwendet. Dies
waren in einer früheren Version die einzigen möglichen Quellen. Durch
Refactoring wurde die Handhabung der Quellen in der Software verallgemeinert,
was diesen Teil modularisiert hat und es in Zukunft auch erlaubt, vergleichsweise einfach andere
Quellen zu nutzen.\\
Die Ergebnisse der Matchings lassen sich von der Ergebnisseite in einem RDF
im XML Format speichern. Da während der Arbeit bemängelt wurde, dass beim
Vergleich verschiedener Ergebnisse und in anderen Matching Tools üblichweise nur
die URIs direkt gezeigt werden, aber nicht die Labels, wurden dem Ausgabeformat
die Label hinzugefügt. Zusätzlich wurde statt dem normalerweise verwendeten
\textit{relation} Tag, ein \textit{connectionType} Tag verwendet, in dem das
konkrete Property steht, z.B. \textit{owl:sameAs}.\\
Nach der ersten Evaluation hat sich gezeigt, dass intuitiv offensichtlich ein
anderer Ansatz gewählt wurde, als beim Ontology Matching sonst üblich. Die
meiste Matching Software versucht, Äquivalente Elemente in zwei Ontologien zu
finden. Das ergibt Sinn, da dies eine sehr wichtige Verbindung ist, wenn man
zwei Ontologien angleichen möchte. Andere Arten von Verbindungen, wie
Gegenteile, Unterklassen oder Bestandteilbeziehungen sind aber auch wichtige
Angaben über Ontologien. Alle denkbaren Arten von Verknüpfungen zu suchen,
anstatt sich eher auf Gleichheit zu fokussieren, verhilft zwar zu einer
breiteren Möglichkeit an Ergebnissen und damit zur Abdeckung des Wissens
zwischen den Ontologien. Gleichzeitig erschwert es aber auch die Entwicklung von
Algorithmen, aufgrund der gestiegenen Komplexität. Welcher Ansatz sinnvoller
ist, entscheidet sich aber von Fall zu Fall und muss von den möglichen Anwendern
getroffen werden. Ein weiterer Unterschied zu den meisten bisher existierenden
Matching Tools ist der Ansatz, nur Vorschläge zu liefern, anhand derer zwei
Ontologien gematcht werden sollen. Es soll dem Nutzer kein perfekter Match
geliefert werden, sondern es muss stets ein sich in der Domäne auskennender
Experte die Ergebnisse begutachten und sie auf ihre Richtigkeit prüfen.

\section{Free Software/Freie Software}
Die Entscheidung, den Simple Ontology Matcher als freie
Software\footnote{\url{https://www.gnu.org/philosophy/free-sw}} zu veröffentlichen, beruht auf zwei Punkten. Zum einen dient es dem
wissenschaftlichen Gedanken, Wissen frei verfügbar zu machen, damit andere davon
profitieren können. Zum anderen hilft es der Diskussion und Weiterentwicklung,
da das Projekt für jeden einsehbar und benutzbar ist.\\
Wie in Kapitel \ref{subsec:Evaluation 1} Evaluation 1 aufgeführt, wurde bei der
Zielsetzung bzw. dem Verständnis der Matcher ein anderer Ansatz als der
verbreitetere gewählt. Weder der eine noch der andere ist generell besser,
sondern es hängt vom Einsatzgebiet und der Zielstellung ab. Mit freier Software
kann frei darüber diskutiert werden und ein anderer Ansatz für die Matcher bzw.
die gesamte Software gewählt werden.

\cleardoublepage
\pagebreak[4]