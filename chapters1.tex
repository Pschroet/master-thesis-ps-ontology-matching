% ---------------------------------------------------
% ----- Chapters of the template
% ----- for Bachelor-, Master thesis and class papers
% ---------------------------------------------------
%  Created by C. Müller-Birn on 2012-08-17, CC-BY-SA 3.0.
%  Freie Universität Berlin, Institute of Computer Science, Human Centered Computing. 
%
\chapter{Vorhandene Software}
\label{chap:existingSoftware}
		
		Da es bereits Software für den Umgang mit Ontologien und das Matching
		derselben gibt, wird sie an dieser Stelle betrachtet. Zum einen, um das Rad in diesem
		Bereich nicht neu und möglicherweise schlechter zu erfinden, sondern zu
		erweitern und/oder andere Aspekte zu behandeln. Und zum anderen soll von
		bestehenden Tools gelernt werden. Da die Auswahl an möglichen Tools groß ist
		\cite{Ber14}\cite{Shv13}\cite{Eng08}, können nicht alle in diesem Rahmen
		ausführlich vorgestellt werden. Daher erfolgt hier nur eine Auswahl. Reine
		Frameworks wurden bei der Bewertung nicht beachtet, da diese für die
		Zielgruppe nicht technischer Experten nicht geeignet sind, sondern für
		Entwickler von entsprechenden Tools.
		
		\section{AgreementMakerLight}
		\textit{AgreementMakerLight} ist ein automatisiertes Ontologie Matching System,
		das eine Weiterentwicklung der Software AgreementMaker ist und seit Anfang
		2013 entwickelt wird. Der AgreementMakerLight ist Open Source, wird unter der
		Apache Lizenz Version 2.0 veröffentlicht und kann daher frei weiterentwickelt
		und verwendet werden. Es ist ein erweiterbares Framework und will Probleme bei
		umfangreichen Ontologie Matchings durch Effizienz angehen. Hauptsächlich
		finden Techniken Anwendung, die auf Elementebene arbeiten und durch
		domänspezifisches Wissen ergänzt werden.
		Der AgreementMakerLight wird als vorbereitetes Projekt für die Integrated
		Development Environment (IDE)
		Eclipse\footnote{\url{https://github.com/AgreementMakerLight/AML-Project}} und
		als ausführbares Java-Archiv
		(jar)\footnote{\url{https://github.com/AgreementMakerLight/AML-Jar}}
		angeboten. Beides trifft die gewählte Zielgruppe aber nicht.
		Als Projekt ist es eher für Entwickler oder zumindest mit Softwareentwicklung
		vertrauten Personen gerichtet. Um eine jar-Datei auszuführen, ist eine
		Installation von Java nötig, was dem Gedanken des im Rahmen dieser Arbeit zu
		erstellenden Tools widerspricht, unkompliziert und sofort benutzbar zu sein.
		Auch besitzen nicht alle Benutzer, die als Zielgruppe für diese Arbeit
		ausgewählt wurden, entsprechende Rechte auf ihren Arbeitssystemen, um
		zusätzliche Software zu installieren.
		
		\section{PARIS}
		\textit{PARIS}\footnote{\url{http://webdam.inria.fr/paris/}}, kurz für
		\textit{Probabilistic Alignment of Relations, Instances, and Schema}, ist ein
		System, um Ontologien im RDF Format zu matchen. Dabei werden Verbindungen auf
		Instanz- und Klassenebene gesucht und wirken sich aufeinander aus. Matchings
		wird eine Wahrscheinlichkeit zugerechnet, um ohne das Festlegen von Parametern
		zu arbeiten. In Tests mit großen Ontologien erzielt PARIS eine Trefferquote
		von etwa 90\%. PARIS steht unter einer Creative Commons Lizenz, die eine
		kommerzielle Nutzung untersagt und eine Namensnennung der Ersteller
		vorschreibt. Ansonsten ist die Erweiterung und Nutzung aber frei.\\
		PARIS wird neben dem reinen Quellcode als jar-Datei ausgeliefert. Wie beim
		AgreementMakerLight ist dies für die Zielgruppe kein idealer Anwendungsfall.
		Weiterhin wird PARIS über die Kommandozeile gestartet, im Gegensatz zu z. B.
		AgreementMakerLight, was eine zusätzliche Hürde darstellt.
		
		\section{COMA 3.0}
		\textit{COMA 3.0}\footnote{\url{http://dbs.uni-leipzig.de/Research/coma.html}}
		ist Matching Tool für Schemata und Ontologien und wurde am Institut für
		Informatik an der Universität Leipzig entwickelt. Es gibt sowohl eine
		klassische Software- als auch eine Webversion. Beide wurden in Java
		entwickelt. Die Webversion wird als Java Applet angeboten. Die Vorgänger von
		COMA 3.0 hießen COMA und COMA++ und es ist eine Weiterentwickelung von diesen.
		Dafür wurde das Management des Workflows verbessert und Features hinzugefügt,
		darunter Ontologie Matching. Da die im Rahmen dieser Masterarbeit entwickelte
		Software als Webapplikation entwickelt werden soll, wird wegen der Ähnlichkeit
		nur das Java Applet betrachtet. Als Java Applets wird Java Software in Java
		Bytecode bezeichnet, die über einen Webbrowser ausgeführt werden kann. Dafür
		werden in Browsern Plug-Ins benötigt, die in der Regel aber nicht
		vorinstalliert sind. Auch gab es immer wieder Probleme mit Sicherheitslücken
		und „das Java-Plug-in war als einer der größten Türöffner für
		Sicherheitsangriffe bekannt und wurde in Untersuchungen immer wieder als
		Risiko Nummer eins auf den PCs dieser Welt bezeichnet“\cite{Heise16}. Zu
		Beginn des Jahres 2016 wurde von Oracle, dem Hersteller von Java, verkündet, dass diese
		Technologie nicht mehr weiterentwickelt wird.  Weiterhin ist die Portabilität
		über Betriebssysteme nicht unproblematisch, da Browser auf mobilen Geräten,
		die Plug-Ins fast ausnahmslos nicht unterstützen.  Das sind alles Gründe, die
		gegen einen Einsatz im Benutzerkreis sprechen, der im Rahmen dieser
		Masterarbeit angesprochen werden soll.
		
		\section{LODE}
		\textit{LODE}\footnote{\url{http://lode.informatik.uni-mannheim.de/}} (Linked
		Open Data Enhancer) ist ein webbasiertes Tool, um Ontologien zu erkunden, zu matchen und zu erweitern. Es ist Open Source und
		wird innerhalb des Play Frameworks   verwendet. Entwickelt wurde es in der
		Kooperation der Universitäten Mannheim, Washington und Indiana.
		
		\section{HerTUDA}
		\textit{HerTUDA}\footnote{\url{http://www.ke.tu-darmstadt.de/resources/ontology-matching/hertuda}}
		ist ein Ontologie Matching Tool der Knowledge Engineering Arbeitsgruppe des Fachbereichs Informatik an der Technischen Universität
		Darmstadt. Es ist auf schnelles und einfaches Arbeiten ausgelegt und arbeitet
		mit syntaktischen Vergleichen von Strings und Herausfiltern irrelevanter
		Matches.
		
		\cleardoublepage
		\pagebreak[4]