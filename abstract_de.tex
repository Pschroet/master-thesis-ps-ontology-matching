% ---------------------------------------------------
% ----- Abstract (German) of the template
% ----- for Bachelor-, Master thesis and class papers
% ---------------------------------------------------
%  Created by C. Müller-Birn on 2012-08-17, CC-BY-SA 3.0.
%  Freie Universität Berlin, Institute of Computer Science, Human Centered Computing. 
%
\pagestyle{empty}

\subsection*{Zusammenfassung}

Die Speicherung von Wissen und Daten in digitaler Form ist heute zu einem sehr
wichtigen Werkzeug geworden, um kollaborativ und global unabhängig von der Fachrichtung gemeinsam zu arbeiten. Damit diese Daten in einem Kontext stehen und semantisch untereinander und mit anderen Daten verknüpft werden können, werden Ontologien entwickelt. In Ontologien wird dann festgelegt, welche Daten wichtig sind und wie sie untereinander in Beziehung stehen. Üblicherweise werden Ontologien für einen bestimmten Bereich oder ein Gebiet erstellt, z.B. eine Tierart oder eine Büchersammlung. Aber auch allgemeiner gefasste Ontologien wurden u.a. im Rahmen der Dublin Core Metadata Initiative  entwickelt. Diese werden konkret dafür benutzt, jegliche Art von Dokumenten und Webressourcen zu beschreiben.\\
Neben der Speicherung von Wissen finden Techniken und Wissen aus dem Bereichen
des Semantic Web und der Ontologien in anderen Bereichen Anwendung, allen voran
Machine und Deep Learning. Einerseits als Anwendungsfall für das Finden von
Matchings, da es sich gut dafür eignet. Andererseits werden Ontologien für das Speichern gewonnener Informationen verwendet. Damit finden Ontologien in einem aktuell im Fokus stehenden Feld von Wissenschaft und Praxis statt.\\
Da Ontologien, wie erwähnt, Anwendung in zahlreichen Gebieten finden, gibt es
Bedarf an Software, mit der auch technisch nicht-versierte Personen gut arbeiten
können.

%\cleardoublepage