% ---------------------------------------------------
% ----- Introduction of the template
% ----- for Bachelor-, Master thesis and class papers
% ---------------------------------------------------
%  Created by C. Müller-Birn on 2012-08-17, CC-BY-SA 3.0.
%  Last upadte: C. Müller-Birn 2015-11-27 
%  Freie Universität Berlin, Institute of Computer Science, Human Centered Computing. 
%
\chapter{Einleitung}
\label{chap:introduction}

\section{Thematische Einordnung}
In der Informatik bezeichnet eine Ontologie je nach Definition „eine
konzeptuelle Formalisierung von Wissensbereichen (Domänen oder Subdomänen)“
\cite{Bie02} oder auch eine „explizite formale Spezifikation einer gemeinsamen
Konzeptualisierung“ \cite{Hes02}.\\
Beide Formulierungen beinhalten, dass etwas
formal definiert wird. Die erste beschränkt sich auf individuelle
Wissensbereiche, während die zweite Aussage einen Austausch durch eine
gemeinsame Konzeptualisierung verschiedener Beteiligter betont. Anwendung finden
Ontologien in der Aufbereitung von semantischem Wissen für Datenmengen, indem
Konzepte in Strukturen überführt werden. Dadurch wird diesen Datenmengen eine
Semantik gegeben. Das ermöglicht es, die Verbindungen und den „Sinn“ auch bei
automatischer Verarbeitung dieser Daten zu erhalten bzw. zu berücksichtigen.
Insbesondere hilft die Semantik auch dabei, den Kontext von Entitäten in der
Ontologie zu erläutern.\cite{Bie02} Dies schafft eine gemeinsame Grundlage, auf
der Daten behandelt und betrachtet werden. So arbeiten alle Nutzer, die eine
Ontologie verwenden, auch mit demselben Vokabular, was Missverständnisse
verringert. Weiterhin müssen in einem Team, welches Ontologien für die Arbeit
mit Daten einsetzt, nicht alle Mitglieder vollständiges und tiefgründiges Wissen
über die Domäne haben, in der gearbeitet wird.\\
Um die Domäne bzw. den Umfang einer Ontologie zu bestimmen, werden zwei Stufen
definiert. \textit{Top Level Ontologien} beziehen sich auf Definitionen, die
unabhängig von spezifischen Domänen Anwendung finden. \cite{Bie02} 
\cite{Hes02}\\
Ein Beispiel dafür ist Dublin
Core\footnote{\url{http://dublincore.org/}}, bei dem alle Arten von Dokumenten
beschrieben werden.\\
\textit{Domain} oder \textit{Task Ontologien} beschäftigen sich mit Begriffen
für einen bestimmten Anwendungsbereich. \cite{Bie02}  \cite{Hes02}\\
Beispielhaft dafür ist die \textit{Music Ontology}, welche „Informationen mit
Bezug zur Musikindustrie“
behandelt.\footnote{\url{https://www.musicontology.com/}}\\
Um Ontologien nicht nur isoliert zu betrachten und da bestimmte Dinge und
Konzepte in mehr als einer Domäne vorzufinden sind, bietet es sich an,
Ontologien zu matchen.\\
Eine Möglichkeit Ontologien und deren Matching zu beschreiben, ist eine
Ontologie mathematisch als ein Paar O = (S, A) zu
definieren. S ist die (ontologische) Signatur, die den Wortschatz
beschreibt. A ist ein Set aus (ontologischen) Axiomen, die die Interpretation
des Wortschatzes in der gewünschten Domäne festlegen. Im Rahmen dieser Definition
ist ein Matching f: S1 -> S2 eine Verknüpfung der Wortschätze
zweier Ontologien O1 = (S1, A1) und O2 = (S2, A2)  unter Beachtung der Struktur
der Signaturen und der Interpretationen. Das Matching einer Ontologie ist dann
mathematisch gesehen ein Morphismus der Signaturen, so dass gilt A2 |= f(A1).
D.h. alle Interpretationen, die die Axiome A2 der Ontologie O2 erfüllen, müssen
auch für die übersetzten Axiome f(A1) von Ontologie O1 gelten. \cite{Hoo14}\\
Unter dem Begriff Konzept wird im Bereich der Ontologien nahezu alles
verstanden, z. B. Dinge, Gedanken oder Begriffe. \cite{Usc95}\\
Diese Konzepte
dienen dazu, Teile der Realität, Daten oder Strukturen in eben diesen Ontologien
abzubilden. Ein häufig genutzter, alternativer Begriff zu ontology
matching ist \textit{ontology alignment}.

\section{Zielsetzung der Arbeit}
Das Ergebnis dieser Masterarbeit ist eine für technisch nicht-versierte Nutzer
gedachte Software, die dazu dient, Ontologien zu matchen. Das Hauptaugenmerk
soll dabei auf geführter Bedienung liegen. In Hinsicht auf diese Arbeit sind
\textit{nichttechnische Experten} Personen, die sich nicht tiefergehend mit
Technik und Software beschäftigen oder auskennen, weil sie z. B. keine
entsprechende Ausbildung oder kein Studium in einem solchen Fachgebiet
absolviert haben.
Solchen Benutzern soll es ermöglicht werden, Ontologien ohne Einrichtungsaufwand
zu matchen.\\
Dabei sollen dem Nutzer Vorschläge für ein Matching unterbreitet werden. Diese
sollen sich anhand eines Algorithmus ergeben. Dieser Algorithmus
beinhaltet eine oder mehrere Vorgehensweisen, nach denen Ontologien gematcht werden. Der Nutzer
kann dann die vorgeschlagenen Matches akzeptieren, verwerfen und andere hinzufügen.\\
Die Matching-Methoden sollen nicht exklusiv und ausschließlich vorgegeben sein,
sondern es muss möglich sein, sie zu verändern. Weiterhin soll es möglich sein,
andere  Methoden zu verwenden, als die zum Ende der Masterarbeit hin
implementierten und auch Methoden, die nach dem Ende der Masterarbeit entwickelt
werden, sollen Verwendung finden können, soweit möglich. Daher wird die
Architektur der Software modular aufgebaut sein.

\section{Vorgehen bei der Umsetzung}
Um eine Software, die in der Zielsetzung beschrieben wurde, umzusetzen und die
notwendige Funktionalität zu implementieren, werden eine Reihe an Methoden und
Technologien Anwendung finden. Diese werden nachfolgend kurz erläutert.

\subsection{Methodische Umsetzung}
Die schon in der Beschreibung der
Software erwähnten Vorschläge deuten bereits an, dass auch Recommender Systeme
Einzug in die Software finden werden. Der übliche Fokus von Recommender Systemen
auf die Empfehlung von Produkten o.ä. (basierend auf den Daten von Nutzern) wird
allerdings nicht genutzt. Es ist zwar denkbar, die Entscheidungen des oder
anderer Nutzer in spätere automatische Matchings einzubeziehen, aber das
Augenmerk der Software soll nicht direkt darauf liegen, dem Anwender Matchings
zu zeigen, die ihm persönlich gefallen, sondern welche, die sinnvoll sind.
Welche Verbindung bei einem gefundenen Matching vorgeschlagen wird, hängt vom
konkreten Matcher bzw. Algorithmus ab. Da die Software auch über eine graphische
Benutzeroberfläche (GUI) verfügen soll und sich Ontologien und verknüpfte
Datenmengen gut als Graph visualisieren lassen, lassen sich auch Techniken aus
dem Gebiet der Information Visualization integrieren.

\subsection{Technische Umsetzung}
Aufgrund der großen Möglichkeiten, die moderne Webtechnologien bieten, wird das
Frontend in einer solchen umgesetzt. Das hat den Vorteil, dass eine
Unabhängigkeit des Betriebssystems bzw. der Nutzungsplattform gegeben ist. Webstandards wurden in
den letzten Jahren erheblich in ihrer Funktionsvielfalt erweitert und über
verschiedene Browser vereinheitlicht. Weiterhin muss neben einem Browser keine
weitere Software installiert werden, sondern das Tool kann sofort genutzt
werden. Die meisten Browser gibt es außerdem für nahezu alle Betriebssysteme,
egal ob auf Desktop/Laptop oder Mobilgeräten wie Smartphones oder Tablets. Damit
werden die Benutzer mit der Technik bedient, die sie selbst verwenden, ohne
eine Laufzeit Umgebung vorauszusetzen, z. B. die Java Runtime Environment für
Java Programme oder eine Python Installation für bestimmte
Betriebssysteme, wo es nicht vorinstalliert ist.\\
Das Backend wird in Python implementiert und dann durch Django eingebettet, um
als Gesamtstück mit dem Frontend zu kommunizieren. Dadurch sind umfangreiche und
rechenintensive Arbeiten möglich. Das ist zwar theoretisch innerhalb von
Browsern möglich, jedoch eher ungewöhnlich und wird in der Regel durch die Dauer
der Operationen von Browsern als unerwünschtes Verhalten gewertet.

\pagebreak[4]