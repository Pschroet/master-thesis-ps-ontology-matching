% ---------------------------------------------------
% ----- Conclusion of the template
% ----- for Bachelor-, Master thesis and class papers
% ---------------------------------------------------
%  Created by C. Müller-Birn on 2012-08-17, CC-BY-SA 3.0.
%  Freie Universität Berlin, Institute of Computer Science, Human Centered Computing. 
%
\chapter{Ausblick}
\label{chap:outlook}      

\section{Das Matching von mehr als zwei Ontologien}
Aktuell können nur jeweils zwei Ontologien miteinander gematcht werden. Durch
einen einfachen Wrapper, der die Ontologien paarweise miteinander matcht, ließe
sich dieser Nachteil beheben.

\section{Nebenläufigkeit}
Wie in Kapitel \ref{ImplementierteMatcher} Implementierte Matcher erwähnt, ist
bei einigen Matchern die Laufzeit nicht ideal. Aber auch bei den anderen Matchern wäre eine schnellere
Bearbeitung der Ongologien positiv. Daher ist eine angedachte Erweiterung für
die Zukunft, die Matcher bzw.
den Matching Prozess nebenläufig zu gestalten. Dafür eignen sich erfahrungsgemäß
andere Sprachen als Python
besser.\footnote{\url{https://www.chrisstucchio.com/blog/2013/why_not_python.html}}\footnote{\url{http://www.dabeaz.com/python/GIL.pdf}}
Eine mögliche Lösung gegen dieses Problem vorzugehen, wäre es, Matcher Module
außerhalb der Python Module zu implementieren und die Nebenläufigkeit dort unterzubringen.

\section{Free Software}
Die Entscheidung, den Simple Ontology Matcher als freie Software zu
veröffentlichen, beruht auf zwei Punkten. Zum einen dient es dem
wissenschaftlichen Gedanken, Wissen frei verfügbar zu machen, damit andere davon
profitieren können. Zum anderen hilft es der Diskussion und Weiterentwicklung,
da das Projekt für jeden einsehbar und benutzbar ist.\\
Wie in Kapitel \ref{subsec:Evaluation 2} Evaluation 2 aufgeführt, wurde bei der
Zielsetzung bzw. dem Verständnis der Matcher ein anderer Ansatz als der
verbreitetere gewählt. Weder der eine noch der andere ist generell besser,
sondern es hängt vom Einsatzgebiet und der Zielstellung ab. Mit freier Software
kann frei darüber diskutiert werden und ein anderer Ansatz für die Matcher bzw.
die gesamte Software gewählt werden.
