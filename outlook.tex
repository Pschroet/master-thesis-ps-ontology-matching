% ---------------------------------------------------
% ----- Conclusion of the template
% ----- for Bachelor-, Master thesis and class papers
% ---------------------------------------------------
%  Created by C. Müller-Birn on 2012-08-17, CC-BY-SA 3.0.
%  Freie Universität Berlin, Institute of Computer Science, Human Centered Computing. 
%
\chapter{Ausblick}
\label{chap:outlook}      

\section{Das Matching von mehr als zwei Ontologien}
Aktuell können nur jeweils zwei Ontologien miteinander gematcht werden. Durch
Änderung der Matching Tool Chain, ließe sich dieser Nachteil beheben, indem die
Ontologien paarweise miteinander gematcht werden.

\section{Nebenläufigkeit}
Wie in Kapitel \ref{ImplementierteMatcher} \textit{Implementierte Matcher}
erwähnt, ist bei einigen Matchern die Laufzeit nicht ideal. Aber auch bei den anderen Matchern wäre eine schnellere
Bearbeitung der Ontologien positiv. Daher ist eine angedachte Erweiterung für
die Zukunft, die Matcher bzw.
den Matching Prozess nebenläufig zu gestalten. Dafür eignen sich erfahrungsgemäß
andere Sprachen als Python
besser.\footnote{\url{https://www.chrisstucchio.com/blog/2013/why_not_python.html}}\footnote{\url{http://www.dabeaz.com/python/GIL.pdf}}
Eine mögliche Lösung gegen dieses Problem vorzugehen, wäre es, Matcher Module
außerhalb der Python Module zu implementieren und die Nebenläufigkeit dort unterzubringen.

\section{Zwischenergebnisse speichern und laden}
Da beim Matching Prozess abhängig vom Umfang der bearbeiteten Ontologien
potenziell sehr viele Paare ermittelt werden, wäre es hilfreich, die
unbearbeiteten Ergebnisse zwischenspeichern zu können, um die Arbeit später
wiederaufnehmen zu können.\\
Eine denkbare Umsetzung wäre die Stelle in den Ergebnissen auf der
Ergebnisseite zu markieren und alle folgenden Matching Vorschläge separat
exportierbar zu machen.

\section{Mehr Informationen auf der Ergebnisseite}
Für die sichtenden Experten wäre es vorteilhaft, leicht mehr Informationen über
die Elemente zu erhalten, ohne die Ergebnisseite verlassen zu müssen. Daher wäre
es gut, bei Bedarf, z. B. als Tooltip, mehr Informationen, wie eine Beschreibung
oder Attribute zu einem Element anzuzeigen.

\section{Zusätzliche Matcher}
Die im Rahmen dieser Arbeit implementierten Matcher decken einige Möglichkeiten
ab, Ontologien zu matchen. Weitere mögliche Matcher lassen sich aus den in
Kapitel \ref{MatcherKategorien} vorgestellten Methoden entwickeln, wie die
Verknüpfung der Elemente untereinander mittels Properties.\\
Der Levenshtein Distance Matcher findet oft Antonyme, sofern es
"`typische"' Änderungen der Vorsilben sind, z. B. \textit{ascending}
und \textit{descending}. Mit diesem lexikalischem Wissen kann man auch einen
Matcher implementieren, der speziell nach solchen Unterschieden in den Labels
sucht.\\
Darüber hinaus ist es denkbar, für Matcher, die auf Wörterbücher
zurückgreifen, andere als die hier verwendeten zu nutzen, z. B.
Wiktionary\footnote{\url{https://de.wiktionary.org/wiki/Wiktionary:Hauptseite}},
WordNet\footnote{\url{https://wordnet.princeton.edu/}} oder Words
API\footnote{\url{https://www.wordsapi.com/}}. Durch andere Wörterbücher
lassen sich möglicherweise auch Techniken wie \textit{Stemming} in die Matcher
integrieren.\\

\section{Information Visualization und Graphen}
Eine schöne Möglichkeit, die Ergebnisse darzustellen, wäre es, Methoden aus der
Information Visualization zu nutzen und/oder Graphen zu verwenden, um die
Elemente der Matchings zusammen mit der Verbindung darzustellen. Aufgrund der
Komplexität ist dies aber keine triviale Aufgabe.

\cleardoublepage
\pagebreak[4]